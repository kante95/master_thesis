% !TEX root = main.tex

\chapter{Conclusions and outlook}
In this thesis, an optical setup for single ion focusing of a 393 nm laser has been designed, built and tested. The design was based on the setups built in other experiments, but it has been improved to avoid clipping that limited the addressing range. The software Zemax was used to design and check the performance of the design. Optimal lenses for the construction were found with the software. Once the simulation was satisfactory, a test setup was built on an optical table where it has been characterized in terms of focal spot size, polarization capabilities, and stability. Here the smallest waist measured was 2.4 $\mu$m, the switching time of the AOD was in the order of 7-8 $\mu$s, and addressing range >150 $\mu$m. Afterwards, the setup was moved and aligned on the ions, where limited physical access did not allow for such easy checks, but instead more advanced quantum optics experiments have been performed.\par
The setup was used for single photon generation and single qubit manipulation. Both of the purposes have been fulfilled: the photon generation was demonstrated in the experiment in Section \ref{exp:photons}, here a string of three ions was loaded into the trap and the focused laser was aligned with the central ion. A laser pulse triggered the photon generation exclusively from the intended ion. The photon detection probability was $\sim 12\%$, and can be further improved as particular attention was not given to the polarization of the addressed Raman beam, but the system already has the capabilities for precise polarization setting. One method to improve the photon generation efficiency is to change the principle quantization axes, set by permanent magnets, to a direction parallel to the addressed laser direction. \par
Qubit manipulation was carried out in the Ramsey interferometer experiment, here we measured the AC stark shift caused by the 393 nm light by imprinting a phase on the qubit encoded in the 729 nm transition. We were able to rotate individual qubits around the z axis, and flip the state of an individual qubit with a quality of 97\%. Moreover, with this experiment the waist of beam was measured to be $1.2-1.3\,\mu$m over a 15 $\mu$m string of 4 ions, and no addressing error was measured up to $10^{-2}$, for 4 ions separated by 4.7-5.1 $\mu$m. In a neighboring advanced trapped-ion quantum simulation experiment a waist of 1.44 $\mu$m was achieved using the same objective but at 729 nm \cite{hempel}.\par
Improvements are possible in this addressing system: the switching speed can be reduced, by e.g. reducing the beam diameter and moving it closer to the piezo. The waist is limited by the objective aperture, so increasing the size of the objective would help achieving a smaller waist.\par
The next natural step is the generation of sequential photons from different ions and this is currently underway at the moment of this thesis writing. Afterwards, entanglement can also be produced between a single ion and a photon. Once entanglement is achieved different investigation prospects will open. We can investigate sending trains of entangled photons to improve the bandwidth of establishing long-distance entanglement \cite{Krutyanskiy2019}, or investigate distribution of multi-partite photon entanglement over distance, which requires mapping multi-partite entanglement onto the travelling photons \cite{Arenskotter:19}.
