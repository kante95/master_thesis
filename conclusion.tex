% !TEX root = main.tex

\chapter{Conclusions and outlook}
In this thesis works, an optical setup for single ion focusing of 393 nm laser has been designed and built. The design was based on the already successful addressing setups built in other experiments, but it has been improved to avoid clipping that limited the addressing range. The software Zemax was used to simulate, and check the performance of the design. Optimal lenses for the construction were also found with the software. Once the simulation was satisfactory, a test setup was built on an optical table where it has been characterized in terms of performance, polarization capabilities, and stability. Here the smallest waist measured was 2.4 $\mu$m, the switching time of the AOD was in the order of 7-8 $\mu$s, and addressing range >150 $\mu$m. Afterwards, the setup was moved and aligned on the ions, where limited physical access did not allow for such easy checks, but instead more advanced quantum optics experiments have been performed.\\
The setup was intended to be used for single photon generations and single qubit manipulations. Both of the purposes has been fulfilled: the photon generation was demonstrated in the experiment in section \ref{exp:photons}, here a string of three ions was loaded into the trap and the focused laser aligned with the central one. A laser pulse triggered the photon generation exclusively from the intended ion. The photon detection probability was $<15\%$, and can definitely be further improved as particular attention was not given to the polarization, but the system already has the capabilities for precise polarization setting. Permanent magnets are still mounted parallel to the previous Raman laser direction, they can be moved in the new direction to improve photon emission.
 Qubit manipulation was carried out in the Ramsey interferometer experiment, here we measured the AC stark shift caused by the 393 nm light by imprinting a phase on the qubit encoded in the 729 nm transition. State readout of the qubits showed the different final states for different phases due to the 393 nm light. Moreover, with this experiment the waist of beam was measured to be $1.2-1.3\,\mu$m and the addressing error was estimated to have an upper bound of $10^{-2}$.\\
The next natural step is the generation of photons from different ions currently undergoing at the moment of this thesis writing. Afterwards, entanglement can also be produced between a single ion and a photon, once more stabilization improvement on the setup are done. This project has several future development, on the quantum network side, this work represents an improved interface between network and quantum computer, transmission bandwidth has drastically increased, dedicated qubits for networking, storing, and computation can now be created and manipulated. It also opens up to the possibility to create multi-ion-multi-photon states with applications in quantum metrology.
