% !TEX root = main.tex

\chapter{Conclusions and outlook}
In this thesis, an optical setup for single ion focusing of 393 nm laser has been designed, built and tested. The design was based on the setups built in other experiments, but it has been improved to avoid clipping that limited the addressing range. The software Zemax was used to design, and check the performance of the design. Optimal lenses for the construction were found with the software. Once the simulation was satisfactory, a test setup was built on an optical table where it has been characterized in terms of focal spot size, polarization capabilities, and stability. Here the smallest waist measured was 2.4 $\mu$m, the switching time of the AOD was in the order of 7-8 $\mu$s, and addressing range >150 $\mu$m. Afterwards, the setup was moved and aligned on the ions, where limited physical access did not allow for such easy checks, but instead more advanced quantum optics experiments have been performed.\\
The setup was used for single photon generation and single qubit manipulation. Both of the purposes has been fulfilled: the photon generation was demonstrated in the experiment in Section \ref{exp:photons}, here a string of three ions was loaded into the trap and the focused laser aligned with the central one. A laser pulse triggered the photon generation exclusively from the intended ion. The photon detection probability was $<15\%$, and can be further improved as particular attention was not given to the polarization of the addressed Raman beam, but the system already has the capabilities for precise polarization setting. Permanent magnets are still mounted parallel to the previous Raman laser direction, they can be moved in the new direction to increase the photon collection efficiency.
 Qubit manipulation was carried out in the Ramsey interferometer experiment, here we measured the AC stark shift caused by the 393 nm light by imprinting a phase on the qubit encoded in the 729 nm transition. State readout of the qubits showed the different final states for different phases due to the 393 nm light. Moreover, with this experiment the waist of beam was measured to be $1.2-1.3\,\mu$m over a 15 $\mu$m string of 4 ions, and no addressing error was measured up to $10^{-2}$, for 4 ions separated by 4.7-5.1 $\mu$m.\\
The next natural step is the generation of sequential photon from different ions and this is currently undergoing at the moment of this thesis writing. Afterwards, entanglement can also be produced between a single ion and a photon. Once entanglement is achieved different investigation prospects will open. We can investigate sending trains of entangled photons to improve the bandwidth of establishing long-distance entanglement \cite{Krutyanskiy2019}, or investigate distribution of multi-partite photon entanglement over distance, which requires mapping multi-partite entanglement onto the travelling photons \cite{Arenskotter:19}.
