% !TEX root = main.tex

\chapter{Conclusions and outlook}
In this thesis works, an optical setup for single ion focusing of 393nm laser has been designed and built. The design was based on the already successful addressing setups built in other experiments, but it has been improved to avoid clipping that limited the addressing range. The software Zemax was used to simulate, and check the performance of the design. Optimal lenses for the construction were also found with the software. Once the simulation was satisfactory, a test setup was built on a spare optical table where it has been characterized in terms of performance, polarization capabilities, and stability. Here the smallest waist measured was 2.4 $\mu$m, the switching time with the AOD were in order of 7-8 $\mu$s, addressing range should be >150 $\mu$m and the setup showed to be stable for at least one hour. Afterwards, the setup was moved and aligned with the experiment, where limited physical access did not allow for such easy checks, but instead more advanced quantum optics experiment could have been performed.\\
The setup was intended to be used for single photon generations and single qubit manipulations. Both of the purposes has been filled: the photon generation was demonstrated in
experiment in section \ref{exp:photons}, here a string of three ions was loaded into the trap and the focused laser aligned with the central one. A laser pulse triggered the photon generation exclusively from the intended ion as we can see from the excitation probability. The photon detection probability was low $<15\%$, and can definitely be further improved. Qubit manipulation was carried out in the Ramsey interferometer experiment, here we measured the AC stark shift caused by the 393nm light by imprinting a phase on the qubit encoded in the 729nm transition. State readout of the qubit showed the different final states for different phases. Moreover, with this experiment the waist of beam was measured to be $1.2-1.3\,\mu$m and the addressing error to have an upper bound of $10^{-3}$.\\
The setup can still be optimized, during the experiments, particular attention was not given to the polarization, but the system already has the capabilities for precise polarization setting. Permanent magnets are still mounted parallel to the previous Raman laser direction, they have to be moved in the new direction.\\
The next natural step is the generation of photons from different ions which requires all ions to be coupled to the cavity vacuum standing wave, a non trivial problem. Entanglement can also be produced between a single ion and a photon, once more stabilization improvement on the setup are done.\\
This project has several future development, on the quantum network side, this work represents an improved interface between network and quantum computer, transmission bandwidth has drastically increased, dedicated qubits for networking, storing, and computation can now be created and manipulated. It also opens up to the possibility to create multi-ion-multi-photon states with applications in quantum metrology.
