% !TEX root = main.tex

\chapter{Conclusions and outlook}
In this thesis works, an optical setup for single ion focusing of 393nm laser has been designed and built.

The setup was intended to be used for single photon generations and single qubit manipulations. Both of the purposes has been filled: the photon generation was demonstrated in
experiment in section \ref{exp:photons}, here a string of three ions was loaded into the trap and the focused laser aligned with the central one. A laser pulse triggered the photon generation exclusively from the intended ion as we can see from the excitation probability. The photon detection probability was low $<15\%$, and can definitely be further improved. Qubit manipulation was carried out in the Ramsey inteferometer experiment, here we measured the AC stark shift caused by the 393nm light by imprinting a phase on the qubit encoded in the 729nm transition. State readout of the qubit showed the different final state for different phases. Moreover, with this experiment the waist of beam was measured to be $1.2-1.3\,\mu$m and the addressing error to have an uppperbound of $10^{-3}$.

The setup can still be optimized, during the experiments, polarization was not set correctly even if the system has the capabilities. Permanent magnets are still mounted parallel to the previous Raman laser, they have to be moved in the new direction.
This project has several future development,
